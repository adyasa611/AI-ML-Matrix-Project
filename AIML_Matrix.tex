\documentclass{beamer}
\usetheme{Boadilla}

\title{AIML Matrix Assignment}
\subtitle{Using Beamer}
\author{Adyasa Mohanty and Krati Arela}
\institute{IIT Hyderabad}
\date{\today}

\begin{document}

\begin{frame}
\titlepage
\end{frame}

\begin{frame}
\frametitle{Question}
The sides of a rhombus ABC are parallel to the lines 
\linebreak (1 -1)x + 2 = 0
\linebreak(7 -1)x + 3 = 0
\linebreak\ If the diagonals of the rhombus intersect at P = (1,2) and the vertex A (different) from the origin is on the y-axis, then find the ordinate of A. 
\end{frame}

\begin{frame}[fragile]
\frametitle{Code}
\begin{semiverbatim}
import numpy as np
import matplotlib.pyplot as plt
A = np.array([1,-1])
B = np.array([7,-1])
C = np.linalg.norm(A)
D = np.linalg.norm(B)

E = np.array([(A[0]/C),(A[1]/C)])
F = np.array([(B[0]/D),(B[1]/D)])
G = E-F
H = E+F
slope1 = (G[1]/G[0])
slope2 = (H[1]/H[0])

POI = np.array([1,2])
x = np.linspace(0.,5.)
\end{semiverbatim}
\end{frame}

\begin{frame}[fragile]
\begin{semiverbatim}
fig,ax = plt.subplots()
ax.plot(x,(slope1*(x-(POI[0]))+POI[1]),'-o',markersize=10,markevery=x.size)
ax.plot(x,(slope2*(x-(POI[0]))+POI[1]),'-o',markersize=10,markevery=x.size)

ax.set_xlim((0.,5.))
ax.set_ylim((0.,5.))

ax.xaxis.set_ticks(np.arange(0.,5.,0.5))
ax.yaxis.set_ticks(np.arange(0.,5.,0.5))

plt.show()
point1 = np.array([0,(slope1*(0-(POI[0]))+POI[1])])
point2 = np.array([0,(slope2*(0-(POI[0]))+POI[1])])



\end{semiverbatim}
\end{frame}

\begin{frame}[fragile]
\begin{semiverbatim}

if point1[1] != 0:
\linebreak \tab \tab	print(point1)
else:
\linebreak \tab \tab	print(point2)
	

\end{semiverbatim}
\end{frame}


\begin{frame}
\frametitle{Solution}
Answer : (0,2.5)
\end{frame}

\begin{frame}
\frametitle{Plot}
\begin{figure}
\includegraphics[height=150pt,length=\paperlength, width=\paperwidth]{Figure_2}
\end{figure}
\end{frame}


\begin{frame}
Thank You!
\end{frame}





\end{document}

