\documentclass{beamer}
\usetheme{Boadilla}
\usepackage[english]{babel}
\usepackage[utf8]{inputenc}
\usepackage{amsmath}
\usepackage{graphicx}
\usepackage[colorinlistoftodos]{todonotes}

\title{AIML Assignment}
\author{Adyasa and Krati}
\institute{IIT Hyderabad}
\date{\today}

\begin{document}


\begin{frame}
\titlepage
\end{frame}

\begin{frame}

\frametitle{Question}
Find the length of diameter of the circle which touches the X-axis at the point
$A=\begin{pmatrix}
1\\
0\end{pmatrix}$ and passes through the point $B=\begin{pmatrix}
2\\
3\end{pmatrix}$

\end{frame}

\begin{frame}\center
\frametitle{Solution}

As the circle touches X-axis at $A = \begin{pmatrix}
1\\
0\end{pmatrix}$
\linebreak
\linebreak X-axis is tangent for the circle
\linebreak
\linebreak Equation of tangent at A:
\linebreak
\linebreak $ \begin{pmatrix}
0 & 1\end{pmatrix}\textbf{x} = 0$
\linebreak
\linebreak This gives equation of normal at A as:
\linebreak
\linebreak $\begin{pmatrix}
1 & 0\end{pmatrix}\textbf{x} = 1$

\end{frame}


\begin{frame}\center


Equation of a circle :
\linebreak
\linebreak $X^TX -2C^TX = R^2 - C^TC$
\linebreak where X is a vector, C represents center and R is radius of 
circle
\linebreak 
\linebreak Given that above circle passes through $A=\begin{pmatrix}
2\\
3\end{pmatrix} and B=\begin{pmatrix}
1\\
0\end{pmatrix}$
\linebreak Substituting A and B in equation of circle,
\linebreak $\begin{pmatrix}
2 & 3\end{pmatrix} \begin{pmatrix}
2\\
3\end{pmatrix} - 2C^T \begin{pmatrix}
2\\
3\end{pmatrix} = R^2 - C^TC$
\linebreak $\begin{pmatrix}
1 & 0\end{pmatrix} \begin{pmatrix}
1\\
0\end{pmatrix} - 2C^T \begin{pmatrix}
1\\
0\end{pmatrix} = R^2 - C^TC$
\linebreak
\linebreak On Simplification, we get

\end{frame}

\begin{frame}\center

$13 - 2C^T\begin{pmatrix}
2\\
3\end{pmatrix} = R^2 - C^TC$
\linebreak
\linebreak $1 - 2C^T\begin{pmatrix}
1\\
0\end{pmatrix} = R^2 - C^TC$
\linebreak
\linebreak Subtracting the above eqautions, we get
\linebreak $12 - 2C^T\begin{pmatrix}
1\\
3\end{pmatrix} = 0$
\linebreak $C^T\begin{pmatrix}
1\\
3\end{pmatrix} = 6$
\linebreak
\linebreak $ \begin{pmatrix} 
1 & 3\end{pmatrix} C = 6 $....(1)
 

\end{frame}

\begin{frame}\center

A normal passes through centre of a circle
\linebreak Hence C passes through equation
\linebreak
\linebreak $\begin{pmatrix}
1 & 0\end{pmatrix} C = 1 $....(2)
\linebreak
\linebreak From (1) and (2), 
\linebreak
\linebreak $ C = \begin{pmatrix}
1 & 0\\
1 & 3\end{pmatrix}^{-1} \begin{pmatrix}
1\\
6\end{pmatrix}$
\linebreak
\linebreak $ C = \begin{pmatrix}
1\\
5/3\end{pmatrix} $


\end{frame}


\begin{frame}\center

Radius of circle :
\linebreak $ R = ||C-A||$
\linebreak $ R = ||\begin{pmatrix}
1\\
5/3\end{pmatrix} - \begin{pmatrix}
1\\
0\end{pmatrix}||$
\linebreak $ R = ||\begin{pmatrix}
0\\
5/3\end{pmatrix}|| $
\linebreak which gives R = 5/3
\linebreak
\linebreak Diameter d = 2R
\linebreak
\linebreak Thus, diameter of required circle is 10/3


\end{frame}


\begin{frame}\center
\frametitle{Figure}
\begin{figure}
\includegraphics [height = 200pt]{Final}
\end{figure}
\end{frame}




\end{document}}
